%----------------------------------------------------------------------------------------
%	Inställningar och dokumentkonfiguration
%----------------------------------------------------------------------------------------

\documentclass[paper=a4, fontsize=11pt]{report} % A4-sida och 11 punkters fontstorlek

\usepackage[T1]{fontenc} % 8-bitarskodning som har 256 glyfer
\usepackage[swedish]{babel} % Svenskt språk
\usepackage[utf8]{inputenc} % För svenska tecken
\usepackage{dtklogos} % Logos
\usepackage{wallpaper} % Bakgrundsbild
\usepackage{fancyhdr} % Specialsidhuvud och sidfot
\usepackage{enumerate} 
\usepackage{xifthen}% provides \isempty test
\pagestyle{fancyplain} % Använd sidhuvud och sidfot på alla sidor
\newcounter{tmpc}
\fancyhead[L]{Seminarie 4 -- 1DV416 -- HT13 -- Windowsadministraion I} % Titel till vänster i sidhuvud
\fancyhead[C]{} % Tomt i mitten
\fancyhead[R]{} % Tomt till höger
\fancyfoot[L]{} % Tomt till vänster
\fancyfoot[C]{} % Tomt i mitten
\fancyfoot[R]{\thepage} % Sidnumrering till höger i sidfoten
\renewcommand\thesection{\arabic{section}} % Section beter sig som i dokumentklassen article

\newcommand{\win}[1]{Microsoft Windows Server\ifthenelse{\isempty{#1}}{}{ #1}}
\newcommand{\gui}[0]{``Server with a GUI''}
\newcommand{\core}[0]{Windows Server Core}
%----------------------------------------------------------------------------------------
%	TITLE SECTION
%----------------------------------------------------------------------------------------
\newcommand\BackgroundPic{
    \put(-50,-50){
    \includegraphics[keepaspectratio,scale=0.65]{lnu_etch.png} % Bakgrundsbild
    }
}
\newcommand\BackgroundPicLogo{
    \put(15,700){
    \includegraphics[keepaspectratio,scale=0.10]{logo.png} % Logga i vänstra hörnet
    }
}

\newcommand{\horrule}[1]{\rule{\linewidth}{#1}} % Skapa hortisontell linje

\title{	\vspace{-10cm}
    \normalfont \normalsize
    \textsc{Linnéuniversitetet} \\ [25pt] % Universitetes namn
    \horrule{0.5pt} \\[0.4cm] % Tunn linje högst upp
    \huge Seminarie 4\\ % Arbetes titel
	\large \textcolor{gray}{1DV416 -- Windowsadministraion I}
    \horrule{0.5pt} \\[0.4cm] % Tunn linje längst ner
}

\author{Jacob Lindehoff} % Författarnas namn

\date{\normalsize\today} % Dagens datum

\begin{document}
\AddToShipoutPicture*{\BackgroundPic} % Lägger in backgrundsbild på första sidan
\AddToShipoutPicture*{\BackgroundPicLogo}
\maketitle % Skriv ut titeln
\noindent % Tabba inte in på första meningen

%------------------------------------------------
%	Introduktion
%------------------------------------------------
\section{Introduktion}
Under detta seminarium kommer vi att titta närmar på följande ämnen:
\begin{itemize}
\item Active Directory
\begin{itemize}
    \item A G/U DL P
    \item Group Policy
\end{itemize}
\item Filsystem
\item Rättigheter
\item Diskhantering
\end{itemize}

%------------------------------------------------
%	Deadline
%------------------------------------------------
\section{Dealine}
Seminariet är den {\color{red}19:e december 2013} och det är obligatorisk närvaro.
Frånfälle måste meddelas i förväg och redovisning sker i from av en skriftlig rapport.
Rapporten ska lämnas in senast {\color{red}3 dagar} efter seminariet och bestå detaljerade svar av samtliga seminariefrågor i löpande text.
\newpage
%------------------------------------------------
%	Seminariefrågor
%------------------------------------------------
\section{Seminariefrågor}
\subsection{Filsystemet NTFS}
\begin{enumerate}
\begin{large}
\item Vilken extra funktionalitet får du med NTFS jämnfört med FAT32?
\item Förklara hur rättighetsarv fungerar i NTFS?
\item Vad är en ACL och vad innehåller den?
\item Vad menas med kumulativa rättigheter?
\item Vad händer med NTFS-rättigheterna när du
\begin{enumerate}[a.]
	\item flyttar en fil inom samma volym
	\item kopierar en fil inom samma volym
	\item flyttar en fil till en annan volym
	\item kopierar en fil till en annan volym
\end{enumerate}
\item Kan mer än en användare vara ägare av en fil eller katalog, motivera ditt svar?
\item Vilka är skillnaderna på NTFS-rättigheter och Utdelnings-rättigheter?
\begin{enumerate}[a.]
	\item Varför finns utdelningsrättigheter när vi kan använda NTFS-rättigheter istället?
\end{enumerate}
\item På vilka två sätt kan Windows hantera feltolerant disklagring?
\item Förklara följande disklagringsmetoder, både hur de fungerar och när det rekommenderas att använda:
\begin{enumerate}[a.]
	\item Simple
	\item Spanned
	\item Striped
	\item Mirrored
	\item RAID-5
\end{enumerate}
\item Vad menas med NTFS-monterade enheter?
\item Har NTFS stöd för Symboliska länkar som finns i t ex Linux?
\setcounter{tmpc}{\theenumi}
\end{large}
\end{enumerate}

\subsection{Access Control}
\begin{enumerate}
\setcounter{enumi}{\thetmpc}
\begin{large}
\item Vad är A G/U DL P strategin, beskriv i detalj och gärna med några exempel?
\setcounter{tmpc}{\theenumi}
\end{large}
\end{enumerate}

\subsection{Group Policy}
\begin{enumerate}
\begin{large}
\setcounter{enumi}{\thetmpc}
\item Vad ändvänds Group Policy till?
\item Vad kan du applicera GPOer på?
\item Vilka AD-objekt påverkas av Group Policy?
\item Kan man applicera ett GPO på flera olika ställen?
\item När appliceras ”Group Policy” inställningarna?
\item Vad gäller vid konflikter mellan gruppolicy-inställningar?
\item Vad menas man att man blockerar GPO arvet på ett OU?
\begin{enumerate}[a.]
	\item Hur påverkad det om man har satt Enforced (No Override) på ett GPO?
\end{enumerate}
\item Vad är Group Policy Security Filtering?
\begin{enumerate}[a.]
	\item Varför bör man undvika detta?
\end{enumerate}
\item Vad används följande verktyg till?
\begin{enumerate}[a.]
	\item gpupdate
	\item gpresult
\end{enumerate}
\item Välj ut 10 olika Group Policy inställningar som du tycker är intressanta och beskriv dessa
\end{large}
\end{enumerate}
\end{document}
